\section{Semiclassical Theory of Conduction II}
We have our configurational space $\v{k}, \v{r}$. For some volume in this space, as we go infinitesimal we have:
\begin{equation}
    g_n(\v{k}, \v{r}, t) \frac{d\v{k}d\v{r}}{4\pi^3} = dN
\end{equation}
where $g_n$ is the density of states. In equilibrium, we have that:
\begin{equation}
    g_n \to g_0 = \frac{1}{e^{\beta(\e_\v{k} - \mu)} + 1}.
\end{equation}

If we recall the Drude model, there we looked at a toy model of electrons travelling through a wire, with equation of motion:
\begin{equation}
    \dod{\v{p}}{t} = \v{F} - \frac{\v{p}}{\tau}
\end{equation}
where the probability of a collision in an infinitesimal time interval $dt$ given by $dp = \frac{dt}{\tau}$ with $\tau$ the relaxation time and $\int dp = 1$. 

\subsection{Non-equilibrium Distribution Function}
 With this picture in mind, we have that the number of electrons emerging from each collision after time $dt$ is:
\begin{equation}
    dg_n = \frac{dt}{\tau(\v{k}, \v{r})}g_0(\v{k}, \v{r}, t)\frac{d^3kd^3r}{4\pi^3}
\end{equation}
where the $g_0(\v{k}, \v{r}, t)\frac{d^3kd^3r}{4\pi^3}$ are the number of electrons that appear in the volume.

Now, let us assume that $g_0(\v{k}, \v{r}, t)$ is unaltered by collisions, that $g_0$ does not depend on $g$ (and as above, $g \to g_0$ in equilibrium). The trajectory of an electron is described by $\v{r}(t), \v{k}(t)$. 

Define $P(t, t')$ as the number of electrons that make it to $\v{k}(t), \v{r}(t)$ without a collision from $\v{k}(t'), \v{r}(t')$. Note that some electrons will leave this trajectory due to collisions.

We then have that:
\begin{equation}
    dN = g(\v{k}, \v{r}, t)\frac{d^3kd^3r}{4\pi^3}= \int_{-\infty}^t P(t, t')\frac{dt'}{\tau(\v{k}(t'), \v{r}(t'))}g_0(\v{k}(t'), \v{r}(t'), t)\frac{d^3kd^4r}{4\pi^3}
\end{equation}

It is worth noting some terminology now - $g$ is the non-equilibrium distribution function, while $g_0$ is the equilibrium distribution function.

Now, we have:
\begin{equation}
    P(t, t') = P(t, t' + dt')\left(1 - \frac{dt'}{\tau(t')}\right)
\end{equation}
so in the limit where $dt' \to 0$:
\begin{equation}
    \dpd{P(t, t')}{t'} = \frac{P(t, t')}{\tau(t')}
\end{equation}
which has solution:
\begin{equation}
    P(t, t') = e^{-\int_{t'}^t \frac{dt''}{\tau(t'')}}
\end{equation}
and we note that $P(t, t) = 1$ (of course, as there are no collisions).

Using this, we can write the non-equilibrium distribution function as:
\begin{equation}
    \begin{split}
        g(t) &= \int_{-\infty}^t dt' g_0(t')\frac{P(t, t')}{\tau(t')}
        \\ &= \int_{-\infty}^t dt' g_0(t')\dpd{P(t, t')}{t'}
        \\ &= \left.g_0(t')P(t, t')\right|_{-\infty}^t - \int_{-\infty}^t P(t, t') \dpd{}{t'}g_0(t') \quad \text{(Integration by parts)}
        \\ &= g_0(t)P(t, t) - g_{0}(-\infty)P(t, -\infty) - \int_{-\infty}^t P(t, t')\dpd{g_0(t')}{t'}
    \end{split}
\end{equation}
Now noting that $P(t, t) = 1$ and $P(t, -\infty) = 0$ (as no electrons from $-\infty$ make it to $t$):
\begin{equation}
    g(t) = g_0(t) - \int_{-\infty}^t dt' P(t, t')\dod{g_0(t')}{t}
\end{equation}
Now, with $g_0(\v{k}, \v{r}) = \frac{1}{e^{\beta(\e_\v{k} - \mu)} + 1}$, and so $g_0(\v{k}(t), \v{r}(t)) = \frac{1}{e^{\beta(\v{r}(t))\left(\e(\v{k}(t)) - \mu(\v{r}(t))\right)} + 1}$ (where this corresponds to ``local'' occupation assuming ``local thermodynamic equilibrium''), we compute with the chain rule that:
\begin{equation}
    \dod{g_0}{t} = \dpd{g_0}{\e_\v{k}}\dpd{\e_{\v{k}}}{\v{k}}\dod{\v{k}}{t} + \dpd{g_0}{\mu}\dpd{\mu}{\v{r}}\dod{\v{r}}{t} + \dpd{g_0}{T}\dpd{T}{\v{r}}\dod{\v{r}}{t}
\end{equation}
Now, recalling that:
\begin{equation}
    \begin{split}
        \dod{\v{r}}{t} &= \v{v}_n(\v{k}) = \frac{1}{\hbar}\nabla_\v{k}\e_\v{k}
        \\ \hbar\dod{\v{k}}{t} = -e(\v{E} + \frac{\v{v}}{c} \times \v{H})
    \end{split}
\end{equation}
we find:
\begin{equation}
    \dpd{g_0}{\e_\v{k}}\dpd{\e_{\v{k}}}{\v{k}}\dod{\v{k}}{t} = \dpd{f}{\e}\v{v} \cdot e(\v{E} + \frac{\v{v}}{c} \times \v{H}) = \dpd{f}{\e}\v{v} \cdot e\v{E}
\end{equation}
where the second term vanishes as $\v{v} \times \v{H}$ is perpendicular to $\v{v}$. For the other terms:
\begin{equation}
    \dpd{g_0}{\mu}\dpd{\mu}{\v{r}}\dod{\v{r}}{t} = -\dpd{f}{\e}\nabla \mu \cdot \v{v}
\end{equation}
\begin{equation}
    \dpd{g_0}{T}\dpd{T}{\v{r}}\dod{\v{r}}{t} = -\dpd{f}{\e}\left(\frac{\e - \mu}{T}\right)\nabla T \cdot \v{v}
\end{equation}
So therefore the non-equilibrium distribution function can be written as:
\begin{equation}
    g(t) = g_0(t) + \int_{-\infty}^t dt'P(t, t')\left[-\dpd{f}{\e}\v{v} \cdot \left(-e\v{E} - \nabla \mu - \frac{\e - \mu}{T} \nabla T\right)\right]
\end{equation}
where $P(t, t') = e^{-(t-t')/\tau(\e_\v{k})}$ assuming $\tau = \tau(\e_\v{k})$. 

Note that last lecture we wrote down a current as:
\begin{equation}
    \v{j} = -e\int_{\e < \e_F} \frac{dk}{4\pi^3}\v{v}(\v{k})
\end{equation}
with the equilibrium distribution (Fermi-Dirac distribution) we now generalize this to:
\begin{equation}
    \v{j} = -e\int \frac{d^3k}{4\pi^3}\v{v}(\v{k}) \cdot g(\v{k})
\end{equation}
where $g(\v{k})$ corresponds to some other, messier distribution.