\section{Bosons, Bose-Einstein Condensation, and Helium-4}
\subsection{Housekeeping}
Reading assignment: Fermi liquid theory in Aschroft and Mermin p345-357. We don't have quite the mathematical tools to explain it, but the excerpt above gives a short qualitative introduction.

Midterm and Final: MT on First week of November. Both exams have the same structure; short in class portion (10-15 minutes), questions concerning things we are expected to know (e.g. $r_s = 2-6$ in metals, or temperature resistivity in Fermi liquid theory is $T^2$ etc.). Then there is a takehome portion with homework-level questions.

\subsection{Boson types and statistics}

But now we switch gears from fermions to bosons. There are two types of bosonic particles we consider in our discussion:
\begin{enumerate}[(a)]
    \item ``Real'' (number-conserving): $\phantom{i}^4\text{He}, \text{Rb}, \text{Na}, \text{K}, \ldots$
    \item ``Emergent'': Phonons (quanta of lattice vibrations), Magnons, \ldots 
\end{enumerate}
The main difference is that for the first class, the number of particles is conserved; the number of Helium atoms in a closed container is fixed. On the other hand, phonons can be created and destroyed out of thin air, so to speak. So there is no conservation law for them.

An important distinction between bosons and fermions is the type of statistics they obey. Mathematically, this is based on commutation relations; in terms of second quantization, bosons obey the commutation relations:
\begin{equation}
    \begin{split}
        &[a_\v{k}^\dag, a_\v{k'}] = \delta_{\v{k}\v{k}'}
        \\ &[a_\v{k}^\dag, a_\v{k'}^\dag] = [a_\v{k}, a_\v{k'}] = 0.
    \end{split}
\end{equation}
Recall the Pauli exclusion principle for fermions (which could be derived from the anticommutation relations); no such principle exists for bosons, of which any number are free to occupy the same quantum state. These commutation relations give rise to the BE distribution function:
\begin{align*}
    \bar{n}_\v{k} = \frac{1}{e^{\beta(\e_{\v{k}} - \mu)} - 1}
\end{align*}
where compared to the FD distribution, the sign in the denominator has flipped. 

\subsection{Deriving the Bose-Einstein Distribution}
Let's review how this distribution function comes about. Consider a non-interacting system:
\begin{equation}
    H = \sum_{\v{k}}(\e_{\v{k}} - \mu)a^\dag_{\v{k}}a_\v{k}
\end{equation}
The thermal average of any operator $\hat{O}$ is given by:
\begin{equation}
    \begin{split}
        \avg{\hat{O}}_\beta &= \frac{1}{Z}\sum_j \bra{j}\hat{O}\ket{j}e^{-\beta (E_j - \mu N)}
    \\  Z &= \sum_j e^{-\beta (E_j - \mu N)}.
    \end{split}
\end{equation}
where the matrix elements of $\hat{O}$ are weighted by the boltzmann factor, of which the partition function $Z$ is the sum of:
Note we use the grand canonic ensemble so we have the $\mu N$. We then have:
\begin{equation}
    \begin{split}
        \avg{\hat{n}_\v{k}} &= \frac{1}{Z}\sum_j \bra{j}a^\dag_{\v{k}}a_{\v{k}}e^{-\beta(\hat{H} - \mu\hat{N})}\ket{j}
        \\ &= \frac{1}{Z}\Tr(a_\v{k}^\dag a_\v{k} e^{-\beta(\hat{H} - \mu\hat{N})})
        \\ &= \frac{1}{Z}\Tr(a_\v{k} e^{-\beta(\hat{H} - \mu\hat{N})}a_\v{k}^\dag)
        \\ &= \frac{1}{Z}\Tr(a_\v{k} a_\v{k}^\dag e^{-\beta(\hat{H} - \mu\hat{N})}e^{-\beta(\e_{\v{k}} - \mu)})
        \\ &= \frac{1}{Z}\Tr((1 + a_\v{k}^\dag a_\v{k}) e^{-\beta(\hat{H} - \mu\hat{N})}e^{-\beta(\e_{\v{k}} - \mu)})
        \\ &= (1 + \avg{\hat{n}_\v{k}})e^{-\beta(\e_{\v{k}} - \mu)}
    \end{split}
\end{equation}
where in the first line, we take advantage of the fact that when $\hat{H}$ acts on its eigenstate, it gives back the eigenvalue. So we can pull the constant into the matrix element and convert the energy into the Hamiltonian. The second line we cast this expression as a trace. In the third line we use the cyclicity of the trace. In the fourth line we skip some math (but it can be done in detail) but this can be viewed as creating a single boson with energy $\e_{\v{k}}$. In the fifth line we use the bosonic commutation relations. In the sixth line we evaluate the expression. We can then solve for $\avg{\hat{n}_\v{k}}$ to obtain:
\begin{equation}
    \avg{\hat{n}_\v{k}} = \frac{1}{e^{\beta(\e_{\v{k}} - \mu)} - 1}.
\end{equation}
Note that a similar derivation can be done for Fermions to get the Fermi-Dirac distribution.

\subsection{Bose-Einstein Condensation}
BE Condensation occurs in real (or number conserving) bosons, most famously Helium-4 at low temperature. The easiest way to see this occurs is to consider the total number of bosons $N$:
\begin{equation}
    N = \sum_\v{k}\bar{n}_\v{k} = \sum_{\v{k}}  \frac{1}{e^{\beta(\e_{\v{k}} - \mu)} - 1}, \quad \e_{\v{k}} = \frac{\hbar^2\v{k}^2}{2m}
\end{equation}
Note that for real bosons $\mu \leq 0$; otherwise we would have $\bar{n}_{\v{k}} < 0$ for some $\v{k}$ which is forbidden. This implies that:
\begin{align*}
    e^{\beta(\e_{\v{k}} - \mu)} \geq e^{\beta\e_{\v{k}}}
\end{align*}
For this reason, we can bound the total number of bosons from above:
\begin{equation}
    \begin{split}
        N &\leq \sum_{\v{k}}\frac{1}{e^{\beta \e_{\v{k}}} - 1}
        \\ &= N_0 + \frac{\Omega}{(2\pi)^3}\int_0^\infty dk \frac{4\pi k^2}{e^{\beta\hbar^2k^2/2m} - 1}
    \end{split}
\end{equation}
where we have separated the sum into two terms; the $\v{k} = 0$ term (which is problematic as it formally diverges; it only comes about as we have discarded the chemical potential) and the rest of the sum rewritten as an integral, which we call $N'(T)$. $\Omega$ here is the volume. We leave the $N_0$ term for now and evaluate the $N'(T)$ by substitution. We let $x = \beta\frac{\hbar^2k^2}{2m}$ and $dx = \beta\frac{\hbar^2}{m}kdk$ so:
\begin{equation}
    N'(T) = \frac{\Omega}{(2\pi)^3}4\pi \sqrt{2}\left(\frac{m}{\beta\hbar^2}\right)^{3/2}\int_{0^+}^\infty \frac{\sqrt{x}dx}{e^x - 1} = C\Omega T^{3/2}
\end{equation}
The actual integral on the right is a finite constant (formally it can be evaluated by considering the Riemann zeta function), but we are really only interested in the temperature dependence, so we've lumped things into a constant $C$. It's also important that $N'(T)$ is extensive/grows proportionally to the system volume. Now we look at how BE condensation comes about from this. We rewrite the inequality as:
\begin{align*}
    N \leq N_0 + N'(T)
\end{align*}
We plot $N'(T)$ as a function of $T$. $N$ is fixed. There is some $T_c$ for which $N'(T)$ intersects $N$; above $T_C$, we have $N_0 = 0$, and below $T_c$ we have $N_0 > 0$. We have an extensive number of bosons in the $\v{k} = \v{0}$ state. This is what is known as BE condensation. In particular as we take $T \to 0$ all of the bosons occupy the ground state. This is not surprising; in the view of QM, if we try to minimize the energy of a set of bosons, we can just cram them all into the ground state (there is nothing preventing us from doing this; no Pauli exclusion)! However in terms of classical physics this phenomena was unusual.

\begin{figure}[htbp]
    \centering
    \textcolor{blue}{TODO-Plot!}
    \caption{<caption>}
    \label{<label>}
\end{figure}

\subsection{Bogoliubov Theory of Helium-4}
This is a classic theory; 1946 (but still valid to this day)! Helium-4 was interesting from the early days of physics at it had interesting superfluid properties. Bogoliubov started off this explanation, and Landau later would give an argument for why Helium-4 is superfluidic (which we cover next lecture).

We consider weakly interacting (spinless) bosons described by the Hamiltonian:
\begin{equation}
    H = \sum_{\v{k}} \e_{\v{k}}a^{\dag}_{\v{k}}a_\v{k} + \frac{1}{2}\sum_{\v{k}\v{p}\v{q}}V_{\v{q}}a^\dag_{\v{k} - \v{q}}a^{\dag}_{\v{p} + \v{q}}a_\v{p}a_\v{k}
\end{equation}
where $V_\v{q}$ is a Fourier transform of a short-range interatomic potential; it is a short range interaction that dies off quickly outside of the Helium atom (on the order of an Angstrom). As usual, $\e_{\v{k}} = \frac{\hbar^2\v{k}^2}{2m}$. We assume (following Bogoliubov) that $V_{\v{q}}$ is weak and $T \ll T_c$. Then we expect the ground state to be close to a perfect BEC, where:
\begin{equation}
    \ket{\Phi^N_0} = (a^\dag_\v{0})^N\ket{0}
\end{equation}
Now we perform the following approximation:
\begin{equation}\label{eq-Napproxbogo}
    \begin{split}
        &a^\dag_\v{0}a_\v{0} \to \avg{a^\dag_\v{0}a_\v{0}} = N_0 \approx N
        \\&a^\dag_\v{0}a^\dag_\v{0} \to N_0
    \end{split}
\end{equation}
The first line: whenever we see the number operator, we take it to be its average value, which is large/close to the total number. The second line is less obvious, but consider $a_\v{0}^\dag a_{\v{0}}^\dag \ket{\Phi_0^N} = \sqrt{(N_0 + 1)(N_0 + 2)}\ket{\Phi_0^{N+2}}$ and we can then take $\sqrt{(N_0 + 1)(N_0 + 2)} \approx N_0$ for $N_0$ large. The assumption being made is that the Hamiltonian is always acting on a state close to the perfect BEC ground state, so we can approximate these operators as we have above. In the interaction term, we split all sums as:
\begin{align*}
    \sum_{\v{k}} = \sum_{\v{k} = 0} + \sum_{\v{k}}'
\end{align*}
and retain only terms containing at least only one power of $N_0$. This is a bit of a mess (as we have 8 sums to work with... ); we will not go through it explicitly, but we justify this approximation by saying that since $N_0$ is large, all terms without powers of $N_0$ are relatively small and hence can be neglected. The result is as follows:
\begin{equation}
    H \approx \sum_{\v{k}}\e_{\v{k}}a^\dag_\v{k}a_\v{k} + \frac{1}{2}N_0^2 V_0 + N_0V_0\sum_{\v{k}}' a_\v{k}^\dag a_\v{k} + N_0\sum_\v{q}'V_\v{q}a^\dag_{\v{q}}a_\v{q} + \frac{1}{2}N_0\sum_\v{q}' V_\v{q}(a_\v{q}a_{-\v{q}} + a_\v{q}^\dag a_{-\v{q}}^\dag)
\end{equation}
The first (kinetic energy) term remains unchanged. The second term comes from $\v{k} = v{q} = \v{p} = \v{0}$, The third term comes from $\v{p} = \v{q} = \v{0}$ or $\v{k} = \v{q} = \v{0}$ and so on. To simplify, we define $\eta_\v{k} = N_0V_\v{k}$ and $\hbar \Omega_\v{k} = \e_\v{k} + \eta_\v{k}$. Notice also that $N_0 + \sum_\v{k}' a^\dag_\v{k}a_\v{k} \approx N$. where $N_0, N \gg N' = \sum_\v{k}' a^\dag_\v{k}a_\v{k}$. With this, let us combine some terms:
\begin{align*}
    \frac{1}{2}N^2V_0 = \frac{1}{2}V_0\left[N_0^2 + 2N_0\sum_\v{k}' a^\dag_\v{k}a_\v{k} + \ldots \right]
\end{align*}
Hence we can write the entire Hamiltonian as:
\begin{equation}
    H \approx \frac{1}{2}N^2 V_0 + \sum_{\v{k}}\left[\hbar \Omega_\v{k}a^\dag_\v{k}a_\v{k} + \frac{1}{2}\eta_\v{k}(a_\v{k}a_{-\v{k}} + a^\dag_\v{k}a^\dag_{-\v{k}})\right]
\end{equation}
We draw our attention to the last term(s); these are ``anomalous terms'', which \emph{do not} conserve particle number. This is a consequence of the Bogoliubov approximation. Physically, $a_\v{k}a_{-\v{k}}$ represent bosons $(\v{k}, -\v{k})$ ``disappearing'' into the condensate. The number of bosons in the condensate is so large that you do not have to keep track of the bosons in the condensate itself; we only need to keep track of the other particles as they disappear and appear out of it.

Bogoliubov also developed a theory of how to treat such Hamiltonians. They can be diagonalized by means of Bogoliubov transformations. We go through what these are next day. We also go through Landau's argument for superfluidity.
